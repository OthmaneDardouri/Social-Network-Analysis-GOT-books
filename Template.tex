\PassOptionsToPackage{unicode}{hyperref}
\PassOptionsToPackage{hyphens}{url}
%
\documentclass[12pt, a4paper]{article}
\usepackage[a4paper,margin=1in]{geometry}
\setlength\parindent{0pt}
\usepackage{mathptmx}
\usepackage{amsmath,amssymb}
\usepackage[T1]{fontenc}
\usepackage[utf8]{inputenc}
\usepackage{textcomp}

\author{Name1 Surname1, Degree Programme, ID number (matricola)
\\Name2 Surname2, Degree Programme, ID number (matricola)}
\date{}
\title{Project Title}

\begin{document}
\maketitle

\section{Introduction}
\label{introduction}

The context includes: the general field (e.g., literature, history,
archaeology, tourism, biology, forensics, religious studies); the
specific application (e.g., literary analysis, quantitative history,
genetics, virology, forensics intelligence, tourism planning, biblical
quantitative studies).

\section{Problem and Motivation}
\label{problem-and-motivation}

What are the problems you want to address? Why are those problems
important (impact, theoretical and/or practical needs, etc.)? What are
the main contributions of the project?

\section{Datasets}
\label{datasets}

How did you gather the data? Did you digitise it? How? Is the material
publicly available? What tools did you use 1) to handle (store,
manipulate) the data and 2) to compute measures on the data?

\section{Validity and Reliability}
\label{validity-and-reliability-not-needed-for-the-project-proposal}

How closely does the model of your dataset represent reality (validity)?
How does the way you treat the data affect the reproducibility of the study (reliability)?

\section{Measures and Results}
\label{measures}

What measures did you apply (brief explanation of how they work)? How do
they relate to the intent of the study? Why are they relevant? What is the connection among the gathered data, the applied measures,
and the properties found?

\section{Conclusion}
\label{conclusion}

Qualitative analysis of the quantitative findings of the study.

\section{Critique}
\label{critique}

Do you think your work solves the problem presented above? To which
extent (completely, what parts)? Why? What could you have done
differently to answer your research problems (e.g., gather data with
additional information, build your model differently, apply alternative
measures)?

\end{document}
